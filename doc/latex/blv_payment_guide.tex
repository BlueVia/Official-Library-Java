\hypertarget{blv_payment_guide_call_abstract_sec}{}\section{Abstract}\label{blv_payment_guide_call_abstract_sec}
The Bluevia Payment API is a set of functions that enables individual, atomic payments based on input economic information. This guide represents a practical introduction to developing applications in which the user wants to provide the Bluevia Payment functionality.\hypertarget{blv_payment_guide_blv_payment_api_making_payment_session_sec}{}\section{Bluevia Payment API: making payment sessions}\label{blv_payment_guide_blv_payment_api_making_payment_session_sec}
A payment session allows to charge a precise, economic amount, on the user's account.\hypertarget{blv_payment_guide_payment_client_basics_sec}{}\subsection{Payment client basics}\label{blv_payment_guide_payment_client_basics_sec}
A Payment client represents the client side in a classic client-\/server schema. This object wraps up the underlying REST client side functionality needed to perform requests against a REST server.

Any Bluevia client performs requests and receives responses in a secure mode. Clients are authorized following the OAuth protocol. This security protocol enables users to grant third-\/party access to their resources without sharing their passwords. So clients store the authorization data -\/called security credentials-\/ to grant access to the server.

Bluevia Payment API uses an extension of OAuth protocol to guarantee secure payment operations. For each payment the user makes he must complete the OAuth process to identify itself and get a valid access token. These tokens will be valid for 48 hours and then will be dismissed. Following sections describe what security credentials we are talking about and the modifications this API needs.\hypertarget{blv_payment_guide_creating_a_payment_client_sec}{}\subsection{Creating a Payment client: BVPayment class}\label{blv_payment_guide_creating_a_payment_client_sec}
The first step in using the Payment client is to create a BVPayment object. As we mentioned earlier this object could have three different working modes: i) as a client which sends requests to a real server, ii) a client working with a real server in testing mode, or iii) as a client which sends requests and receives responses to/from a sandbox server.\hypertarget{blv_payment_guide_paymentclient_features_working_modes_sec}{}\subsubsection{Payment client features: working modes}\label{blv_payment_guide_paymentclient_features_working_modes_sec}
When you create the client, you can specify various working modes: 
\begin{DoxyItemize}
\item BVBaseClient.Mode.LIVE \par
In the Live environment your application uses the real network, which means that you will be able to send real transactions to real Movistar, O2 and Vivo customers in the applicable country.


\item BVBaseClient.Mode.TEST \par
The Test mode behave exactly like the Live mode, but the API calls are free of chargue, using a credits system. You are required to have a Movistar, O2 or Vivo mobile number to get this monthly credits.


\item BVBaseClient.Mode.SANDBOX \par
The Sandbox environment offers you the exact same experience as the Live environment except that no traffic is generated on the live network, meaning you can experiment and play until your heart’s content. 
\end{DoxyItemize}\hypertarget{blv_payment_guide_paymentclient_features_security_credentials_sec}{}\section{Payment API OAuth protocol}\label{blv_payment_guide_paymentclient_features_security_credentials_sec}
Bluevia uses OAuth as its authentication mechanism which enables websites and applications to access the Bluevia API's without end users disclosing their personal credentials. In order to grant access to the server any client has to be created passing the security credential as parameter in its constructor. These security credentials are managed internally and added as a HTTP header in every request sent to the server. Such Oauth security credentials are:


\begin{DoxyItemize}
\item {\bfseries Consumer key} \par
The string identifying the application-\/ you obtained when you registered your application within the provisioning portal.


\item {\bfseries Consumer Secret} \par
A secret -\/a string-\/ used by the consumer to establish ownership of the consumer key. 
\end{DoxyItemize}

To authorize each payment operation a new valid access token has to be retrieved, completing this special OAuth process.\hypertarget{blv_payment_guide_paymentclient_features_code_example}{}\subsection{Payment client features: code example}\label{blv_payment_guide_paymentclient_features_code_example}
Find below an example about how to create a Payment client taking into account all information previously given. This snippet shows how to access a Bluevia server using OAuth security credentials


\begin{DoxyCode}
// Mode.LIVE indicating the client works against a real Bluevia server.
try {
        BVPayment paymentClient = new BVPayment(Mode.LIVE, "consumer_key", "consu
      mer_secret");
} catch (BlueviaException e){
        log.error(e.getMessage());
}
\end{DoxyCode}
\hypertarget{blv_payment_guide_paymentclient_features_freeing_resources_sec}{}\subsection{Payment client features: freeing resources}\label{blv_payment_guide_paymentclient_features_freeing_resources_sec}
It is very important to free the resources that the client instantiates to work. To do this call the close method after finishing using the client:


\begin{DoxyCode}
paymentClient.close();
\end{DoxyCode}
\hypertarget{blv_payment_guide_retrieving_payment_request_token_sec}{}\subsection{Step 1: retrieving a request token}\label{blv_payment_guide_retrieving_payment_request_token_sec}
First, you have to retrieve a request token to be authorised by the user. In this case you have to use the getPaymentRequestToken function, which includes the Payment info besides the usual request tokens params:


\begin{DoxyCode}
RequestToken requestToken = getPaymentRequestToken(int amount, String currency, S
      tring name, String serviceId) throws BlueviaException
\end{DoxyCode}
\hypertarget{blv_payment_guide_retrieving_payment_verification_sec}{}\subsection{Step 2: Take the user to Bluevia Connect}\label{blv_payment_guide_retrieving_payment_verification_sec}
Take the user to Bluevia Connect to authorise the application as usual.\hypertarget{blv_payment_guide_retrieving_payment_access_token_sec}{}\subsection{Step 3: retrieveing an access token}\label{blv_payment_guide_retrieving_payment_access_token_sec}
With the obatined oauth\_\-verifier, you can now get the accessToken from the user using the function:


\begin{DoxyCode}
OAuthToken getPaymentAccessToken(String oauthVerifier, String token, String secre
      t) throws BlueviaException
\end{DoxyCode}


Note: the Payment client will store the access token internally to make calls to payment API using the same client. Additionally the access token is returned to to allow storing it persistentely for subsequent requests (remember the lifetime of the token is 48 hours). The are two ways in order to use a previosly obtained token: creating a new BVPayment instance supplying it or using the setAccessToken function of the client.\hypertarget{blv_payment_guide_paymentclient_features_funcitions}{}\section{Payment client features: making a payment}\label{blv_payment_guide_paymentclient_features_funcitions}
\hypertarget{blv_payment_guide_paymentclient_features_payment_function}{}\subsection{PaymentClient features : payment function}\label{blv_payment_guide_paymentclient_features_payment_function}
It's a once shot function. Once used for an OAuthToken, given in the client constructor, it cannot be used again to make a payment (but it can be used to retrieve the status of the operation). If the result status of the process of payment is a success then a transaction identifier is returned. Otherwise, if the result status of the payment process is a failure then none is returned and a BlueviaException with detailed information is thrown. This function provides the main feature of the API, the making of payments.

It is possible to receive status notifications by providing Bluevia with an URI where notifications must be sent:


\begin{DoxyCode}
PaymentResult payment(int amount, String currency, String endpoint, String correl
      ator) throws BlueviaException, IOException
\end{DoxyCode}


Params:\par
 {\bfseries amount} : Amount to be charged, it may be an economic amount or a number of 'virtual units' (points, tickets, etc) (mandatory). \par
 {\bfseries currency} : Type of currency which corresponds with the amount above, following ISO 4217 (mandatory). \par
 {\bfseries endpoint} : The endpoint to receive notifications of payments operations (optional). \par
 {\bfseries correlator} : An string to correlate the requests and the notifications (optional). \par
 Return:\par
 {\bfseries paymentResult} : A result containing the identifier assigned by the server for the transaction and other information like the status.\par
 Throws:\par
 An BlueviaException informs about errors encountered in the request/response.\par
 An IOException informs about connectivity errors.\par
\hypertarget{blv_payment_guide_paymentclient_features_get_status_function}{}\subsubsection{Payment client features : get payment status}\label{blv_payment_guide_paymentclient_features_get_status_function}
The function getPaymentStatus allows to request the status of a previous payment operation. If the refund process is unsuccessful then an BlueviaException with detailed information is thrown.

Use its transaction identifier as sole parameter.


\begin{DoxyCode}
PaymentStatus getPaymentStatus(String transactionId) throws BlueviaException, IOE
      xception
\end{DoxyCode}
\hypertarget{blv_payment_guide_paymentclient_features_cancel_authorization_function}{}\subsubsection{Payment client features : cancel authorization function}\label{blv_payment_guide_paymentclient_features_cancel_authorization_function}
After having requested a token you can cancel the authorization for that token before making the payment, or cancel get status requests after the payment. If the cancel authorization process is unsuccessful then an BlueviaException with detailed information is thrown.


\begin{DoxyCode}
void cancelAuthorization() throws BlueviaException, IOException
\end{DoxyCode}
\hypertarget{blv_payment_guide_paymentclient_processes_code_example}{}\subsubsection{Bluevia Payment API: code examples}\label{blv_payment_guide_paymentclient_processes_code_example}

\begin{DoxyCode}
// Make a payment session
BVPayment paymentClient = null;

try {

        // 1. Create the client (you have to choose the mode and include the OAut
      h authorization values. No access token is needed)
        paymentClient = new BVPayment(Mode.LIVE, "consumer_key", "consumer_secret
      ");

        // 2. OAuth process
        RequestToken requestToken = paymentClient.getPaymentRequestToken(100, "GB
      P", "bluevia", "service_id");
        //...
        //...

        // Authorization in Bluevia Connect Portal
        // Get OAuth verifier from GUI
        String oauth_verifier = null;

        //...
        //...

        OAuthToken accessToken = paymentClient.getPaymentAccessToken(oauth_verifi
      er, requestToken.getToken(), requestToken.getSecret());

        // 3. Make the payment
        PaymentResult result = paymentClient.payment(100, "GBP");
        System.out.println("Successful payment with transaction identifier: " + t
      ransactionId);


        // 4. Get Payment status
        PaymentStatus status = paymentClient.getPaymentStatus(result.getTransacti
      onId());
        System.out.println("Payment status " + status.getTransactionStatus());          
                
} catch (BlueviaException e) {
        log.error(e.getMessage(), e);
} catch (IOException e) {
        log.error(e.getMessage(), e);
} finally {

        // 5. Remember to close the client to free resources
        if (paymentClient != null)
                paymentClient.close();
} 
\end{DoxyCode}
 